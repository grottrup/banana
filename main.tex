\documentclass{article}
\usepackage[utf8]{inputenc}
\usepackage[table]{xcolor}

\title{Requirements rapport - Softwareteknologi} % Sets article title
\author{Artin Ghalamkary - (AU......) \and Karsten Bak Malle - (AU644054) \and EL PHILLIPE (AU......) \and Nikita Svanholm Alsøer (AU639436)}% Sets authors name
\date{September 2021}

\begin{document}

\maketitle
\section*{Preface}
This should define the expected readership of the document and describe its version history, including a
rationale for the creation of a new version and a summary of the changes made in each version.

\section*{Introduction}
This should describe the need for the system. It should briefly describe the system’s functions and
explain how it will work with other systems. It should also describe how the system fits into the overall
business or strategic objectives of the organization commissioning the software.

\section*{Glossary}
This should define the technical terms used in the document. You should not make assumptions about the
experience or expertise of the readerwz

database
agile process
plan-driven development process
system architecture
system model

\section*{User requirements definition}
Here, you describe the services provided for the user. The nonfunctional system requirements should also
be described in this section. This description may use natural language, diagrams, or other notations that are
understandable to customers. Product and process standards that must be followed should be specified.
\definecolor{cornflowerblue}{rgb}{0.39, 0.58, 0.93}
\begin{center}
\begin{tabular}{| c |}
 \hline
 User requirements\\
 \hline 
 \rowcolor{cornflowerblue}
 1. Mail client should be able to display, send, and receive emails aswell as let the user be able & \rowcolor{cornflowerblue} to reply and forward the emails received through the selected server. \\
 \hline
 \rowcolor{cornflowerblue}
 2. Mail client should provide the user the opportunity to login to their personal email\\
 \hline
 \rowcolor{cornflowerblue}
 3. The user should be able to blacklist any email\\
 \hline
 \rowcolor{cornflowerblue}
 4. The user should be able to login to multiple emails at the same time\\
 \hline
 \rowcolor{cornflowerblue}
 5. The user should be able to delete any email\\
 \hline
 \rowcolor{cornflowerblue}
 6. The user should be able to mark emails as read/unread\\
 \hline
 \rowcolor{cornflowerblue}
 7. The user should be able attach files and pictures to their emails\\
 \hline
 \rowcolor{cornflowerblue}
 8. The user should be able to compose a email\\

\end{tabular}
\end{center}

\begin{center}
\begin{tabular}{| c |}
 \hline
 Nonfunctional requirements\\
 \hline 
 \rowcolor{cornflowerblue}
 1. Limited database \\
 \hline
 \rowcolor{cornflowerblue}
 2. Interval of email fetching\\
 \hline
 \rowcolor{cornflowerblue}
 3. Mail client will be driven by localhost\\
 \hline
 \rowcolor{cornflowerblue}
 4. Mail client downtime\\
 \hline
 \rowcolor{cornflowerblue}
 5. File attachment size constraint\\
 \hline
\end{tabular}
\end{center}

\section*{System architecture}
This chapter should present a high-level overview of the anticipated system architecture, showing the
distribution of functions across system modules. Architectural components that are reused should be
highlighted.

\section*{System requirements specification}
This should describe the functional and nonfunctional requirements in more detail. If necessary,
further detail may also be added to the nonfunctional requirements. Interfaces to other systems may be
defined.

\section*{System model}
This might include graphical system models showing the relationships between the system
components and the system and its environment. Examples of possible models are object models, dataflow models, or semantic data models.

\section*{System evolution}
This should describe the fundamental assumptions on which the system is based, and any anticipated
changes due to hardware evolution, changing user needs, and so on. This section is useful for system
designers as it may help them avoid design decisions that would constrain likely future changes to the
system.

The thought process behind our system has been to create a symbiosis between the plan driven development, as well as the agile process. To prepare system evolution our focus going forward is to be aware of the implementation of the the agile process, since the system will be required to change, due to  as the user needs change as well. 

\section*{Appendices}
es These should provide detailed, specific information that is related to the application being
developed; for example, hardware and database descriptions. Hardware requirements define the
minimal and optimal configurations for the system. Database requirements define the logical
organization of the data used by the system and the relationships between data.

Database will be developed through the usage of the Django framework.

\section*{Index}
Several indexes to the document may be included. As well as a normal alphabetic index, there may be
an index of diagrams, an index of functions, and so on.

\end{document}
